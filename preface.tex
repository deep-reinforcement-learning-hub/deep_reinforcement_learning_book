%%%%%%%%%%%%%%%%%%%%%%preface.tex%%%%%%%%%%%%%%%%%%%%%%%%%%%%%%%%%%%%%%%%%
% sample preface
%
% Use this file as a template for your own input.
%
%%%%%%%%%%%%%%%%%%%%%%%% Springer %%%%%%%%%%%%%%%%%%%%%%%%%%

\preface
\vspace{-\baselineskip}
\vspace{-\baselineskip}
\vspace{-\baselineskip}
\vspace{-\baselineskip}
\vspace{-\baselineskip}
\vspace{-\baselineskip}
\vspace{-\baselineskip}
\vspace{-\baselineskip}
%% Please write your preface here
Reinforcement learning is a branch of machine learning (ML), which is different from traditional machine learning such as supervised and unsupervised learning. 
It focused on learning from agent-environment interactions to achieve certain goal(s) optimally. The learning process is interactive and driven both internally and externally.
Rencent developments in other machine learning techniques, especially neural networks, have been driving forces which help advance this field significantly. 
The improvements in both the scope and the depth of problems being studied and solved in the area are encouraging.
This book starts from introduction to math foundations, to recent improvements and advanced topics.
\vspace{\baselineskip}

\noindent {\large \bfseries  Why This Book}

\noindent This book is written as a comprehensive introduction to the field of reinforcement learning, with the focus on recent improvements and new techniques in the area. 
It starts with the introduction to traditional reinforcment learning and its evolution, math foundations in the area, to the recent technique advances in the area 
that are being applied and developed, including deep reinforcement learning algorithms. Then, one chapter is dedicated to the realistic applications,
followed by advanced topics in both academy and industry and attemptive solutions, which make RL-based models and systems empirical.



\vspace{\baselineskip}
\begin{flushright}\noindent
April, 2024 \hfill {\it Emily Lyon, etc..}\\
\end{flushright}


